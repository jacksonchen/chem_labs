\documentclass[10pt, letterpaper]{article}
\usepackage[top=40pt,bottom=80pt,left=60pt,right=60pt]{geometry}
\usepackage[utf8]{inputenc}
\usepackage{epstopdf}
\usepackage{titling}
\usepackage{amsmath}               % great math stuff
\usepackage{amsfonts}              % for blackboard bold, etc
\usepackage{gensymb}
\usepackage[version=3]{mhchem}
\usepackage{graphicx}
\newcommand{\subtitle}[1]{%
  \posttitle{%
    \par\end{center}
    \begin{center}\large#1\end{center}
    \vskip0.5em}%
}

\begin{document}

\title{Yeast Catalyzed Decomposition Rate Effects of Hydrogen Peroxide}
\author{Jackson Chen}
\subtitle{Fairview High School Period 8 IB Chemistry II, Dan Albritton \\ Semester 2 IB Internal Assessment Part 1}
\date{22 January 2015}

\maketitle



\section{Problem}
How does varying the amount of yeast catalyst affect the rate of decomposition of hydrogen peroxide H$_2$O$_2$?

\section{Use of Data}
To measure the rate of decomposition of H$_2$O$_2$, the experiment will be performed in an open container where the O$_2$ product will be allowed to escape. This will decrease the mass of the solution, and thus it can be found how much O$_2$ gas was produced over a certain period of time since the only loss of mass is due to the O$_2$ gas. Once the mass difference is calculated, the moles of O$_2$ gas produced can be calculated. Taking this value over the total volume of the solution gives the change in concentration of O$_2$. Once the rate of production of O$_2$ is known, the rate of decomposition of H$_2$O$_2$ can be calculated with stoichiometry.

\begin{equation}
2\text{H}_2\text{O}_2 \rightarrow 2\text{H}_2\text{O} + \text{O}_2
\end{equation}

From Equation 1, the rate of production of O$_2$ is one half the rate of decomposition of H$_2$O$_2$.

\section{Variables}
\begin{itemize}
\item{The mass of yeast added to the H$_2$O$_2$ is the independent variable and it is quantitative.}
\item{The volume of O$_2$ gas produced is the dependent variable and it is quantitative. This volume will be used in accordance with time to calculate the rate.}
\item{The initial concentration of H$_2$O$_2$ is a controlled variable and it is quantitative.}
\item{The temperature of the system and its surroundings is a controlled variable and it is quantitative.}
\item{The type of catalyst added to the H$_2$O$_2$ is a controlled variable and it is qualitative.}
\item{The pressure of the system and its surroundings is a controlled variable and it is quantitative.}
\item{The time for the reaction is a controlled variable and is quantitative.}
\end{itemize}

\section{Apparatus
}
\begin{itemize}
\item{Electronic mass balance (if not electronic, then the balance should have 0.1 gram graduations and range from 0 to 500 grams)}
\item{Erlenmeyer flask: Up to 50 mL}
\item{Graduated Cylinder: Up to 10 mL with .1 mL graduations}
\item{Yeast}
\item{Weighing Paper}
\item{Scoopula}
\item{3\% aqueous solution hydrogen peroxide}
\item{Timer (preferably electronic) with 0.01 second increments}
\item{Thermometer: -10 to 100 degrees C with 0.2 degree graduations}
\end{itemize}

\section{Diagram}
\begin{figure}[!htb]
\centering
\includegraphics[scale=1.4]{Diagram.pdf}
\caption{Arrangement of apparatus used in the experiment.}
\label{fig:digraph}
\end{figure}

\section{Procedure}

\subsection{Balanced Equation}
\begin{equation}
2\text{H}_2\text{O}_2 \rightarrow 2\text{H}_2\text{O} + \text{O}_2
\end{equation}

\subsection{Methods}

\begin{enumerate}
\item{Wear goggles.}
\item{Measure 10 mL of H$_2$O$_2$ using the graduated cylinder and pour into Erlenmeyer flask. Record the temperature of the system using the thermometer.}
\item{Measure the mass of the system (Erlenmeyer flask + 10 mL of H$_2$O$_2$) using the electronic mass balance, record it.}
\item{Measure 5 grams of yeast using the mass balance, then add it to the Erlenmeyer flask using a scoop and start the timer}
\item{Wait 10 seconds and then measure the mass of the system (Erlenmeyer flask + solution) using the mass balance and record it}
\item{Repeat step 5 for 10 times}
\item{Repeat steps 2-6 for 3 trials using the same mass of yeast. If the temperature is different from the previous trials, redo step 2.}
\item{Repeat step 7 and use 10 g, 20 g, 40 g of yeast for each set of trials}
\end{enumerate}

\section{Assumptions}
\begin{itemize}
\item{The pressure of the room stays constant throughout the experiment. The experiment will be conducted in the same room at around the same time to avoid major potential changes in the room.}
\item{Any change in temperature of the solution during the experiment should not have any noticeable effects on the results.}
\item{The concentration of hydrogen peroxide will not change throughout the trials. The source of hydrogen peroxide for the experiment will stay the same for all the trials to minimize the impact of this assumption.}
\end{itemize}

\end{document}

